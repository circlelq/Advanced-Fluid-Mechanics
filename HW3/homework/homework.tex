\documentclass[12pt]{article}
\usepackage{natbib}
\usepackage{url}
\usepackage{stmaryrd}
\usepackage{mathrsfs}
\usepackage{amsmath}
\usepackage{graphicx}
\usepackage{parskip}
\usepackage{fancyhdr}
\usepackage{commath}%定义d
\usepackage[UTF8,scheme = plain]{ctex}
\usepackage{geometry}
\usepackage{bm}
\usepackage{siunitx}
\usepackage{float}
\usepackage{subfig}
\usepackage{titlesec}
\usepackage{caption}
\usepackage{paralist}
\usepackage{multirow}
\usepackage{booktabs} % To thicken table lines
\usepackage{diagbox}
\usepackage{authblk}
\usepackage{indentfirst}
\usepackage{amsthm}
\usepackage{fontspec}
\usepackage{color}
%\usepackage{txfonts} %设置字体为times new roman
\usepackage{lettrine}
\usepackage{nameref}
%\usepackage[nottoc]{tocbibind}
\usepackage{amssymb}%font
\usepackage{lipsum}%make test words
\usepackage{picinpar}%words around the picture
\usepackage[all]{xy}%draw arrow
\usepackage{asymptote}%draw picture
\usepackage[perpage]{footmisc}%脚注每页清零
\usepackage{esint}

\geometry{bottom=3cm,left=3cm,right=3cm,top=3cm}
% \footskip = 60pt

% \setmainfont{TimesNewRomanPSMT}
\setsansfont{Helvetica-Light}
\setCJKmainfont[ItalicFont=STKaitiSC-Regular,BoldFont=STSongti-SC-Black]{STSongti-SC-Regular}
\setCJKsansfont[BoldFont=STHeitiSC-Medium]{STHeitiSC-Light}


%\setmainfont{Times New Roman}

\ctexset{today=old}%日期类型设置

% ======================================
% = Color de la Universidad de Sevilla =
% ======================================
\usepackage{tikz}
\definecolor{PKUred}{cmyk}{0,1,1,0.45}

%超链接设置
\usepackage[breaklinks,colorlinks,linkcolor=PKUred,citecolor=PKUred,pagebackref,urlcolor=PKUred]{hyperref}
\usepackage{cleveref}
% \newcommand{\crefpairconjunction}{ 和 }


\newcommand{\hsp}{\hspace{20pt}}
\newcommand{\nhsp}{\hspace{-30pt}}
\titleformat{\section}{\Large\bfseries}{%\arabic{section}
\hspace{-22pt}\textcolor{PKUred}{\vrule width 2pt}\hsp}{0pt}{}


\titleformat{\subsection}
  {\normalfont\large\bfseries}{}{0em}{}

\renewcommand*\footnoterule{%
    \vspace*{-3pt}%
    {\color{PKUred}\hrule width 2in height 0.4pt}%
    \vspace*{2.6pt}%
}


%% Color the bullets of the itemize environment and make the symbol of the third
%% level a diamond instead of an asterisk.
%h\renewcommand*\textbullet{\dag}
\renewcommand*\labelitemi{\color{PKUred}\textbullet}
\renewcommand*\labelitemii{\color{PKUred}--}
\renewcommand*\labelitemiii{\color{PKUred}$\diamond$}
\renewcommand*\labelitemiv{\color{PKUred}\textperiodcentered}



%%% Equation and float numbering
\numberwithin{equation}{section}		% Equationnumbering: section.eq#
\numberwithin{figure}{section}			% Figurenumbering: section.fig#
\numberwithin{table}{section}				% Tablenumbering: section.tab#


%代码设置
\usepackage{listings}
\usepackage{fontspec} % 定制字体
\newfontfamily\menlo{SFMono-Regular}
\usepackage{xcolor} % 定制颜色
\definecolor{mygreen}{rgb}{0,0.6,0}
\definecolor{mygray}{rgb}{0.5,0.5,0.5}
\definecolor{mymauve}{rgb}{0.58,0,0.82}
\lstset{  
numbers=left,
numberstyle=\footnotesize\menlo,
basicstyle=\footnotesize\menlo,
backgroundcolor=\color{white},      % choose the background color
columns=fullflexible,
tabsize=4,
breaklines=true,               % automatic line breaking only at whitespace
captionpos=b,                  % sets the caption-position to bottom
commentstyle=\color{mygreen},  % comment style
escapeinside={\%*}{*)},        % if you want to add LaTeX within your code
keywordstyle=\color{blue},     % keyword style
stringstyle=\color{mymauve}\ttfamily,  % string literal style
frame=single,
rulesepcolor=\color{red!20!green!20!blue!20},
% identifierstyle=\color{red},
language=c++,
xleftmargin=4em,xrightmargin=2em, aboveskip=1em,
framexleftmargin=2em,
numbers=left
}

%脚注
\renewcommand\thefootnote{\fnsymbol{footnote}}

%定义常数i、e、积分符号d
\newcommand\mi{\mathrm{i}}
\newcommand\me{\mathrm{e}}

%%% Maketitle metadata
\newcommand{\horrule}[1]{\rule{\linewidth}{#1}} 	% Horizontal rule
\newcommand{\tabincell}[2]{\begin{tabular}{@{}#1@{}}#2\end{tabular}}


\setcounter{secnumdepth}{2}
\usepackage{bm}
\usepackage{autobreak}
\usepackage{amsmath}
\graphicspath{{fig/}}
\setlength{\parindent}{2em}

%pdf文件设置
\hypersetup{
	pdfauthor={袁磊祺},
	pdftitle={Advanced Fluid Mechanics Homework 3}
}

\title{
		\vspace{-1in} 	
		\usefont{OT1}{bch}{b}{n}
		\normalfont \normalsize \textsc{\LARGE Peking University}\\[1cm] % Name of your university/college \\ [25pt]
		\horrule{0.5pt} \\[0.5cm]
		\huge \bfseries{Advanced Fluid Mechanics Homework 3} \\
		\horrule{2pt} \\[0.5cm]
}
\author{
		\normalfont 								\normalsize
		College of Engineering \quad 2001111690  \quad 袁磊祺\\	\normalsize
        \today
}
\date{}

\begin{document}

%%%%%%%%%%%%%%%%%%%%%%%%%%%%%%%%%%%%%%%%%%%%%%
\captionsetup[figure]{name={图},labelsep=period}
\captionsetup[table]{name={表},labelsep=period}
\renewcommand\contentsname{目录}
\renewcommand\listfigurename{插图目录}
\renewcommand\listtablename{表格目录}
\renewcommand\refname{参考文献}
\renewcommand\indexname{索引}
\renewcommand\figurename{图}
\renewcommand\tablename{表}
\renewcommand\abstractname{摘\quad 要}
\renewcommand\partname{部分}
\renewcommand\appendixname{附录}
\def\equationautorefname{式}%
\def\footnoteautorefname{脚注}%
\def\itemautorefname{项}%
\def\figureautorefname{图}%
\def\tableautorefname{表}%
\def\partautorefname{篇}%
\def\appendixautorefname{附录}%
\def\chapterautorefname{章}%
\def\sectionautorefname{节}%
\def\subsectionautorefname{小小节}%
\def\subsubsectionautorefname{subsubsection}%
\def\paragraphautorefname{段落}%
\def\subparagraphautorefname{子段落}%
\def\FancyVerbLineautorefname{行}%
\def\theoremautorefname{定理}%
\crefname{figure}{图}{图}
\crefname{equation}{式}{式}
\crefname{table}{表}{表}
%%%%%%%%%%%%%%%%%%%%%%%%%%%%%%%%%%%%%%%%%%%

\maketitle

\section{1}

在飞机参考系中,飞行中可用的净推力$ F_N $由两个动量通量之差得出,即
\begin{equation}
	F_{N}= \dot{m}_{\text {air}} V_{j}-\dot{m}_{\text {air}} V.
\end{equation}
其中 $\dot{m}_{\text {air}}$ 为单位时间进入的空气质量,$V_j = 650$m/s 为喷气速度,$V=222$m/s 为进气速度, 8000m 处的空气密度为$\rho = 5.25 \times 10^{-1} \mathrm{kg/m^3}$,
\begin{equation}
	\dot{m}_{\text {air}} = \frac{\pi D^2}{4} \times V \rho ,
\end{equation}
其中$D=0.86$m是进口截面直径,所以
\begin{equation}
	F_{N}= \frac{\pi D^2}{4} \times V \rho \left( V_{j}- V\right) = 2.90 \times 10^{4} \mathrm{N}.
\end{equation}

\section{2}

\subsection{}

空气密度$\rho = 1.29$kg/m³, 摩尔质量$\mu = 29 $g/mol,满足理想气体状态方程
\begin{equation}
	p = \frac{\rho R T}{\mu}
\end{equation}
设船和水之间的空气压强为$p_1$,大气压强为$p_0$,压力差
\begin{equation}
	F = (p_1-p_0)S = 98000 \mathrm{N},
\end{equation}
与重力平衡,其中$S = 3\times 10$m$^2$为软裙覆盖面积, 由伯努利方程,
\begin{equation}
	p_0 + \frac{1}{2}\rho v^2 = p_1,
\end{equation}
所以
\begin{equation}
	v = \sqrt{\frac{2F}{S\rho}} = 71 \si{\metre /\second}.
\end{equation}
气流量
\begin{equation}
	\dot{m} = v \rho h L = 71 \times 1.29 \times 0.02 \times 26 \mathrm{kg/s} = 47\mathrm{kg/s},
\end{equation}
其中$h=20\mathrm{mm}$是间隙,$L=2\times (3+10)\mathrm{m}=26\mathrm{m}$.



\subsection{}

\begin{equation}
	P = \frac{1}{2}\rho v^2 L v =  \num{6.0e+06} \si{\watt} = \num{8.2e+03} \mathrm{hp} .
\end{equation}

\section{3}

我觉得远处的星体对物体的影响可以忽略不计,旋转水桶使得水向下凹陷是因为时空的不对称性,宇宙是有边界的,所以不存在没有物体的完美空间.

在一个远离各个星球(星球对物体对引力极小时),存在一个坐标系,在这个坐标系里,互不影响的物体做匀速直线运动,这个坐标系就是惯性坐标系,任何相对其做匀速直线运动的坐标系都是惯性坐标系,而任何相对其做加速运动(包括旋转)的坐标系都是非惯性系.

爱因斯坦的相对论并没有讨论把所有物体都去掉以后空间的行为.他谈论的是空间与时间的耦合(狭义相对论),以及空间、时间、物质的耦合(广义相对论).讨论的基础都是假设存在一个惯性系.


\section{4}

速度场分布
\begin{equation}
	\bm{v} = \nabla \times \left(\frac{1}{4\pi}\int_\tau \frac{\bm{\omega}(\xi,\eta ,\zeta )}{r} \dif \tau \right) = \frac{1}{4\pi} \nabla \times \oint_L \frac{\Gamma}{r} \dif \bm{l} = - \frac{\Gamma}{4\pi} \int_L \frac{\bm{r}\times \dif \bm{l}}{r^3}
\end{equation}
设$\bm{x_0}=(x_0,y_0,z_0)$处速度场势函数为$0$, 则
\begin{equation}
	\varphi(\bm{x}) = - \int_{\bm{x}_0}^{\bm{x}} \bm{v} \cdot \dif \bm{l} 
\end{equation}
$\bm{l}$沿着$\bm{x_0}$到$\bm{x}$的某一路径,由于速度场成环状,且圈住涡丝,所以势函数是多值的.$\varphi(\bm{x})$ 不同值的物理意义是:势函数的差为绕一圈后圈住的涡丝量乘以负环量$-\Gamma$.

\section{5}

设矢量线$\bm{x} = \bm{x}(s,t)$ 是一条物质线,$s$是定义矢量线的参数.在$t=0$时刻,它与涡线重合的充分必要条件是
\begin{equation}
	\pd{\bm{x}}{s} \times \bm{\omega} = 0 
\end{equation}
或
\begin{equation}
	\pd{\bm{x}}{s} = f\bm{\omega}\quad(t=0),
	\label{eq:52}
\end{equation}
其中$f\not= 0$是一个标量.若涡线在$t>0$时仍与此武陟县相切,则必有
\begin{equation}
	\od{}{t}(\pd{\bm{x}}{s} \times \bm{\omega}) =0. 
	\label{eq:51}
\end{equation}
利用物质线元的物质导数公式
\begin{equation}
	\od{}{t}(\dif \bm{x}) = \dif \bm{x} \cdot \nabla\bm{u} = \dif \bm{u},
\end{equation}
\cref{eq:51} 变为
\begin{equation}
	\od{}{ t}\left(\pd{ \bm{x}}{ s} \times \bm{\omega}\right)=\left(\pd{ \bm{x}}{ s} \cdot \nabla \bm{u}\right) \times \bm{\omega}+\pd{ \bm{x}}{ s} \times \od{ \bm{\omega}}{ t},
	\label{eq:53}
\end{equation}
若涡线在$t>0$仍与$\bm{x}(s,t)$重合,可将\cref{eq:52}代入\cref{eq:53},得
\begin{equation}
	\od{}{ t}\left(\frac{\partial x}{\partial s} \times \omega\right)=f \omega \times\left(\od{ \omega}{ t}-\omega \cdot \nabla u\right)=f \omega \times(\nabla \times a),
\end{equation}
和\cref{eq:51} 相比,克制涡线是物质线的必要条件是
\begin{equation}
	\omega \times(\nabla \times a) =0.
	\label{eq:54}
\end{equation}
反之,若成立,且在$t=0$时有\cref{eq:52},则$\pd{\bm{x}}{s} \times \bm{\omega}$总是0,所以\ref{eq:54}又是涡线为物质线的充分条件.\qed


\section{6}

\begin{equation}
	\begin{aligned}
	\frac{\dif H}{\dif t} &=\frac{\dif}{\dif t} \int_V \bm{u} \cdot \bm{\omega} \dif V \\
	&=2\int _V (\nabla \times \bm{a}) \cdot \bm{u} \dif V
	\end{aligned}
\end{equation}
在旋转参照系中,离心力有势,设刚体旋转角速度为$\bm{\Omega}$,
\begin{equation}
	\bm{a} = - \bm{u} \cdot \nabla \bm{u} - \nabla (h+\phi) - 2 \bm{\Omega} \times \bm{u},
\end{equation}
其中$h=\int \dif p/\rho,\ \phi$ 是体力势.
\begin{equation}
	\begin{aligned}
		\frac{\dif H}{\dif t} &=-4\int _V (\nabla \times (\bm{\Omega} \times \bm{u}) \cdot \bm{u}) \dif V\\
		&=-4 \bm{\Omega} \cdot \int _V (\nabla \bm{u}) \cdot \bm{u}+(\nabla \cdot \bm{u}) \bm{u} \dif V\\
		&=-2 \bm{\Omega} \cdot \int _V \nabla \bm{u}^2 \dif V\\
		&=-2 \bm{\Omega} \cdot \int_{\partial V} \bm{n} \bm{u}^2 \dif S\\
		&=0.
	\end{aligned}
\end{equation}
上式考虑到了$\bm{\Omega}$是常矢量,以及液体不可压,并且边界速度,边界法向涡量为$0$,并且用到了
\begin{equation}
	\nabla \times (\bm{a}\times \bm{b}) = (\bm{b} \cdot \nabla) \bm{a} - (\bm{a}\cdot\nabla)\bm{b}+\bm{a} \nabla \cdot \bm{b} - \bm{b} \nabla \cdot \bm{a}.
\end{equation}

\section{7}

\begin{equation}
	\frac{\dif \boldsymbol{I}_{\partial \mathcal{V}}}{\dif t}=\int_{\partial \mathcal{V}} \frac{\Dif}{\Dif t}(\phi \boldsymbol{n} \dif S), \quad \frac{\dif \boldsymbol{L}_{\partial \mathcal{V}}}{\dif t}=\int_{\partial \mathcal{V}} \frac{\Dif}{\Dif t}(\boldsymbol{x} \times \phi \boldsymbol{n} \dif S)
\end{equation}

\begin{equation}
	\begin{aligned}
		\frac{\dif \boldsymbol{I}_{\partial \mathcal{V}}}{\dif t}=&\int_{\partial \mathcal{V}} \frac{\Dif}{\Dif t}(\phi \boldsymbol{n} \dif S)\\
		=&\int_{\partial \mathcal{V}} \phi\frac{\Dif}{\Dif t}( \boldsymbol{n} \dif S) + \int_{\partial \mathcal{V}} \boldsymbol{n} \dif S\frac{\Dif}{\Dif t}\phi \\
		=&-\int_{\partial \mathcal{V}} \phi \bm{n} \cdot \nabla\nabla \phi \dif S + \int_{\partial \mathcal{V}} \boldsymbol{n} \dif S\frac{\Dif}{\Dif t}\phi \\
		=&-\int_{ \mathcal{V}} \nabla \cdot (\phi \nabla\nabla \phi) \dif V  + \int_{ \mathcal{V}} \nabla \frac{\Dif}{\Dif t}\phi \dif V \\
		=&-\int_{ \mathcal{V}} \phi \Delta \bm{u} +\nabla \frac{\bm{u}^2}{2} \dif V  + \int_{ \mathcal{V}} \frac{\partial \phi }{\partial t} + \nabla\bm{u}^2 \dif V \\
		=& \int_{ \mathcal{V}} \frac{\partial \phi }{\partial t} + \frac{1}{2} \nabla\bm{u}^2 \dif V =- \frac{1}{\rho} \int_{ \mathcal{V}} \nabla p \dif V \\
		=&- \frac{1}{\rho} \int_{\partial \mathcal{V}}  p \bm{n} \dif S \\
	\end{aligned}
\end{equation}

上式在推导过程中没有考虑到远场是否为0.

\section{8}

\subsection{3.12}

\begin{equation}
	\frac{\dif \boldsymbol{I}_{\partial \mathcal{V}}}{\dif t}=\int_{\partial \mathcal{V}} \frac{\Dif}{\Dif t}(\phi \boldsymbol{n} \dif S), \quad \frac{\dif \boldsymbol{L}_{\partial \mathcal{V}}}{\dif t}=\int_{\partial \mathcal{V}} \frac{\Dif}{\Dif t}(\boldsymbol{x} \times \phi \boldsymbol{n} \dif S)
\end{equation}


\begin{equation}
	\begin{aligned}
		\frac{\dif \boldsymbol{I}_{\partial \mathcal{V}}}{\dif t}=&\int_{\partial \mathcal{V}} \frac{\Dif}{\Dif t}(\phi \boldsymbol{n} \dif S)\\
		=&\int_{\partial \mathcal{V}} \phi\frac{\Dif}{\Dif t}( \boldsymbol{n} \dif S) + \int_{\partial \mathcal{V}} \boldsymbol{n} \dif S\frac{\Dif}{\Dif t}\phi \\
		=&-\int_{\partial \mathcal{V}} \phi \bm{n} \cdot \nabla\nabla \phi \dif S + \int_{\partial \mathcal{V}} \boldsymbol{n} \dif S\frac{\Dif}{\Dif t}\phi \\
		=&-\int_{ \mathcal{V}} \nabla \cdot (\phi \nabla\nabla \phi) \dif V  + \int_{ \mathcal{V}} \nabla \frac{\Dif}{\Dif t}\phi \dif V \\
		=&-\int_{ \mathcal{V}} \phi \Delta \bm{u} +\nabla \frac{\bm{u}^2}{2} \dif V  + \int_{ \mathcal{V}} \frac{\partial \phi }{\partial t} + \nabla\bm{u}^2 \dif V \\
		=& \int_{ \mathcal{V}} \frac{\partial \phi }{\partial t} + \frac{1}{2} \nabla\bm{u}^2 \dif V =- \frac{1}{\rho} \int_{ \mathcal{V}} \nabla p \dif V \\
		=&- \frac{1}{\rho} \int_{\partial \mathcal{V}}  p \bm{n} \dif S \\
	\end{aligned}
\end{equation}

\subsection{3.2}

\subsubsection{(1)}

反证法

假设有两个解$\bm{u_1},\ \bm{u_2}$,
\begin{gather}
	\nabla \cdot \bm{u}_{1,2} = \theta,\\
	\nabla \times \bm{u}_{1,2} = \bm{\omega},
\end{gather}
边界上,
\begin{equation}
	\bm{u}_{1} \cdot \bm{e}_n = \bm{u}_{2} \cdot \bm{e}_n,\quad  \text{或者}\quad \bm{u}_{1} \times \bm{e}_n = \bm{u}_{2} \times \bm{e}_n.
\end{equation}
设
\begin{equation}
	\bm{u} = \bm{u}_{1} - \bm{u}_{2},
\end{equation}
则
\begin{gather}
	\nabla \cdot \bm{u} = 0,\\
	\nabla \times \bm{u} = 0.
\end{gather}
边界上,
\begin{equation}
	\bm{u}_{1} \cdot \bm{e}_n = 0,\quad  \text{或者}\quad \bm{u}_{2} \times \bm{e}_n = 0.
\end{equation}
因为
\begin{equation}
	\nabla \times \bm{u} = 0,
\end{equation}
所以
\begin{equation}
	\bm{u}=\nabla \phi.
\end{equation}
又因为
\begin{equation}
	\nabla \cdot \bm{u} = 0,
\end{equation}
所以
\begin{equation}
	\nabla^2 \phi = 0.
\end{equation}
\begin{enumerate}
	\item 若$\bm{u} \cdot \bm{e}_n = 0$,则
	\begin{equation}
		\partial_n \phi = 0.
	\end{equation}
	\item 若$\bm{u} \times \bm{e}_n = 0$,则
	\begin{equation}
		\partial_t \phi = 0.
	\end{equation}
	即$\phi$为常数
\end{enumerate}
对以上两种情况,即为拉普拉斯方程的两种边界条件,由唯一性定理可知,解唯一,即$\phi$为常数,所以
\begin{equation}
	\bm{u} = \nabla \phi = 0.
\end{equation}
所以解唯一.\qed

\subsubsection{(2)}

对于无滑移和无穿透边界条件来说,速度在边界就是为0,按照上面的分析来看,全场速度为0,但是对于有粘度的流体来说,流体在边界有一个边界层,边界层涡量极大,和给定条件有所不同,所以边界条件不矛盾.















\nocite{*}

\bibliographystyle{plain}
\phantomsection

\addcontentsline{toc}{section}{参考文献} %向目录中添加条目,以章的名义
\bibliography{homework}

\end{document}
